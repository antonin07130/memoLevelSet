% -------- %
% Appendix %
% -------- %
%
% \newpage
%
% \pagestyle{fancy}
% \fancyhead{} % Headers
% \fancyhead[LE,RO]{Appendix}
%
% \appendix
% \addcontentsline{toc}{chapter}{Appendix}
%
% \newpage
% \section{Mercer Kernel}
% \label{an:mercer}
%
% \emph{Theorem (Mercer's theorem):} Suppose that $K : \chi \times \chi \mapsto \Rset$ is symmetric and satisfies \\ \mbox{$sup_{x,y} K(x,y) < \infty$}, and define
% \begin{equation}
%   \label{eq:an-mercer-defTk}
%   T_Kf(x) = \int_\chi K(x,y)f(y)dy.
% \end{equation}
% Suppose that $T_K : L^2(\chi) \mapsto L^2(\chi)$ is positive semi-definite; thus,
% \begin{equation}
%   \label{eq:an-mercer-def2}
%   \int_\chi \, \int_\chi K(x,y) f(x) f(y) dxdy \, \leq \,0
% \end{equation}
% for any $f \in L^2(\chi)$. Let $\lambda_i$, $\Psi_i$ be the eigenfunctions and eigenvectors of $T_K$, with
% \begin{equation}
%   \label{eq:an-mercer-def3}
%   \int_\chi K(x,y) \Psi_i(y) dy \, = \, \lambda_i \Psi_i(x).
% \end{equation}
% Then
% \begin{enumerate}
%  \item $\displaystyle \sum_i {\lambda_i }< \infty$
%  \item $\displaystyle sup_x \Psi_i(x) < \infty$
%  \item $\displaystyle K(x, y) = \sum_{i=1}^\infty {\lambda_i \Psi_i(x) \Psi_i(y)}$
% \end{enumerate}
% where the convergence is uniform in x, y.
%
% Such a kernel defines a Mercer kernel. This gives the mapping into feature space as
% \begin{equation}
%   \label{eq:an-mercer-kernel}
%   x \, \mapsto \, \Phi(x) = ( \sqrt{\lambda_1} \Psi_1(x), \, \sqrt{\lambda_2} \Psi_2(x), \cdots)^T
% \end{equation}


