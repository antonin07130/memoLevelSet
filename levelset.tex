\documentclass[twoside, 11pt]{myreport}

%% ----------------------- %%
%% start of the 'preamble' %%
%% ----------------------- %%

%___________________________________________________________
%
% -- Declaration of packages
\usepackage{amsmath} % Maths pour avoir
\usepackage{amssymb} % les ensembles

\usepackage{epsfig} % Pour afficher des figures
\usepackage{subfig} % Pour afficher plusieurs figures cote a cote
\usepackage{wrapfig} % Pour pouvoir ecrire autour des figures

\usepackage{color}	% Pour utiliser les couleurs
\definecolor{red}{rgb}{1,0,0}	% Definition de la couleur rouge
\definecolor{darkblue}{rgb}{0,0,0.2}

% \usepackage[T1]{fontenc}    % Carateres accentues
% \usepackage[frenchb]{babel} % Typographie francaise

\usepackage[top=3cm, bottom=3cm, left=2.5cm, right=2.5cm]{geometry}

\usepackage{fancyhdr} % Pour modifier les en-tete et pied de page
\usepackage{lastpage} % Pour avoir acces au numero de la derniere page

\usepackage{hyperref} % Pour avoir des liens vers les differentes parties
\hypersetup{
		colorlinks = true, % Pour virer les rectangles rouges autour des liens
		linkcolor = darkblue, %black,
		citecolor = darkblue, %black,
		filecolor = darkblue, %black,
		urlcolor = darkblue, %black
}

% -- End of declaration of packages
%___________________________________________________________
%

%___________________________________________________________
%
% -- Various expressions
\newcommand{\todo}[1]{\textbf{\emph{\textcolor{red}{#1}}}} % Definition de la commande todo (met le texte en rouge)
\newcommand{\refeq}[1]{(\ref{#1})} % Pour mettre des parentheses autour des numeros d'equations

\newcommand{\bs}{B-spline}
\newcommand{\eg}{\emph{e.g. }}
\newcommand{\etal}{\emph{et al. }}
\newcommand{\ie}{\emph{i.e. }}
\newcommand{\ls}{level-set}
\newcommand{\Ls}{Level-set}
\newcommand{\LS}{$\phi(\cdot)$}

\newcommand{\vecx}{\mathbf}

\def\Rset{\mathbb{R}}
\def\Zset{\mathbb{Z}}
\def\Nset{\mathbb{N}}
%___________________________________________________________
%
% -- End of various expressions

%%___________________________________________________________
%%
%% -- Definition of headers and footers
\pagestyle{fancy}
\renewcommand{\chaptermark}[1]{\markboth{\thechapter .~#1}{}}

\fancyhead{} % Headers
\fancyhead[LE,RO]{\nouppercase\leftmark}
\fancyhead[LO,RE]{\nouppercase\rightmark}

\fancyfoot{} % Footers
\fancyfoot[CE,CO]{\thepage /\pageref{LastPage}}

\headsep=20pt % Definition of spaces between header and text
\headheight=16pt % Header's size
\renewcommand{\footrulewidth}{0.5pt} % Width of the line separating the header from the text

%% -- Fin de la definition
%%___________________________________________________________
%%

%% ----------------- %%
%% end of 'preamble' %%
%% ----------------- %%

%% -------------- %%
%% begin document %%
%% -------------- %%
\begin{document}
%___________________________________________________________
%
% -- Declaration of each part

% ---------- %
% Title Page %
% ---------- %

\begin{titlepage}
 
\begin{center}
 
 
% Upper part of the page
\vspace*{0.5cm}
\vfill
 
% Title
\hrule 
\vspace*{0.4cm}
\Huge \bfseries {\Ls}
\vspace*{0.4cm} 
\hrule 

\vfill

\end{center}
 
\end{titlepage}

% -------- %
% Sommaire %
% -------- %

\newpage

\pagestyle{fancy}
\fancyhead{} % Headers
\fancyhead[LE,RO]{\nouppercase\leftmark}

\tableofcontents
\addcontentsline{toc}{chapter}{Table of Contents}


\newpage

\pagestyle{fancy}
\fancyhead{} % Headers
\fancyhead[LE,RO]{Introduction}

\chapter*{Introduction}
\label{chap:intro}
\addcontentsline{toc}{chapter}{Introduction}

\vspace{0.5cm}
\hspace{0.7cm} This document gives a short description of some \ls-based algorithms. For further details about these methods, the reader is invited to refer to the cited algorithms. For each method, the energy criterion which is minimized, the derived evolution equation and the main properties of the method are given.
% ---------------- %
% Level-set Method %
% ---------------- %

\newpage

\fancyhead{} % Headers
\fancyhead[LE,RO]{\nouppercase\leftmark}
\fancyhead[LO,RE]{\nouppercase\rightmark}

\chapter{``Classical'' \ls~methods}
\label{chap:ls-methods}

\section[Caselles]{Caselles \cite{Caselles1997}}
\label{sec:Caselles}

\paragraph{Energy criterion}
\begin{equation}
	\label{eq:NRJ_caselles}
	E(\Gamma) = \int_0^1 g(I(\Gamma(q))) \|\Gamma'(q)\| dq,
\end{equation}
where
\begin{equation}
	\label{eq:g_caselles}
	g(I) = \frac{1}{1 + \| \nabla (G \ast I) \|^2},
\end{equation}
$I(\cdot)$ corresponds to the image intensity, $\Gamma$ is the parametric curve and $G$ is a gaussian filter of variance 1.

\paragraph{Evolution equation}
\begin{equation}
	\label{eq:evol_caselles}
	\frac{\partial \phi}{\partial t}(\vecx{x}) = g(I(\vecx{x})) \| \nabla \phi(\vecx{x}) \| \kappa + \nabla g(I(\vecx{x})) \nabla \phi(\vecx{x}).
\end{equation}
where $\kappa = \text{div}\left(\frac{\nabla \phi(\vecx{x})}{\| \nabla \phi(\vecx{x}) \|} \right)$ corresponds to the curvature of the evolving contour.

\paragraph{Properties}

\begin{itemize}
	\item This algorithm is a contour-based method \ie the gradient of the image is used to compute the force function. The curve will thus be driven to regions with high gradient.
	\item This method does not require any regularization term as it is intrinsic to the method.
\end{itemize}


\newpage
\section[Chan \& Vese]{Chan \& Vese \cite{ChanVese2001}}
\label{sec:ChanVese}

\paragraph{Energy criterion}
\begin{equation}
	\label{eq:NRJ_chanvese}
	E(\phi) = \int_\Omega F(I(\vecx{x}), \phi(\vecx{x})) \, d\vecx{x} + \lambda \int_\Omega \delta(\phi(\vecx{x})) \| \nabla \phi(\vecx{x}) \|d\vecx{x},
\end{equation}
where $\delta$ is the dirac function and 
\begin{equation}
	\label{eq:F_chanvese}
	F(I(\vecx{x}),\phi(\vecx{x})) = H(\phi(\vecx{x}))(I(\vecx{x}) - v)^2 + (1 - H(\phi(\vecx{x})))(I(\vecx{x}) - u)^2,
\end{equation}
$H$ is the Heaviside function, $u$ and $v$ are two parameters updated at each iteration as follows:

\begin{equation}
	\label{eq:u_chanvese}
	u = \frac{\int_\Omega (1 - H(\phi(\vecx{x}))) \cdot I(\vecx{x})\, d\vecx{x}}{\int_\Omega 1 - H(\phi(\vecx{x})) \, d\vecx{x}} 
\end{equation}
\begin{equation}
	\label{eq:v_chanvese}
	v = \frac{\int_\Omega H(\phi(\vecx{x})) \cdot I(\vecx{x})\, d\vecx{x}}{\int_\Omega H(\phi(\vecx{x})) \, d\vecx{x}} 
\end{equation}

The first integral of \refeq{eq:NRJ_chanvese} correspond to a data attached term and the second is a regularization term that acts on the evolving contour.

\paragraph{Evolution equation}
\begin{equation}
	\label{eq:evol_chanvese}
	\frac{\partial \phi}{\partial t}(\vecx{x}) = \delta(\phi(\vecx{x})) \nabla_\phi F(I(\vecx{x}), \phi(\vecx{x})) + \lambda \delta(\phi(\vecx{x})) \text{div}\left(\frac{\nabla \phi(\vecx{x})}{\|\nabla \phi(\vecx{x})\|}\right),
\end{equation}
where
\begin{equation}
	\label{eq:gradF_chanvese}
	\nabla_\phi F(I(\vecx{x}),\phi(\vecx{x})) = \delta(\phi(\vecx{x}))((I(\vecx{x}) - v)^2 - (I(\vecx{x}) - u)^2)
\end{equation}

\paragraph{Properties}

\begin{itemize}
	\item This algorithm is a region-based method. It tends to separate the image into two homogeneous region (according to their mean value).
	\item The evolution is only computed on the narrow-band of the \ls~thus making it sensitive to initialization.
\end{itemize}


\newpage
\section[Chunming Li]{Chunming Li \cite{Li2008}}
\label{sec:Li}

\paragraph{Energy criterion}
\begin{eqnarray}
	\label{eq:NRJ_li}
	\nonumber E(\phi) = & \lambda_1 \int \int K_\sigma(\vecx{x} - \vecx{y}) | I(\vecx{y}) - f_1(\vecx{x}) |^2 H(\phi(\vecx{x})) \, d\vecx{y}d\vecx{x} \\
	& + \lambda_2 \int \int K_\sigma(\vecx{x} - \vecx{y}) | I(\vecx{y}) - f_2(\vecx{x}) |^2 (1 - H(\phi(\vecx{x})))\, d\vecx{y}d\vecx{x} \\ 
	\nonumber & + \nu \int| \nabla H(\phi(\vecx{x})) |d\vecx{x} + \mu \int \frac{1}{2}\left(\|\nabla \phi(\vecx{x}) \| - 1\right)d\vecx{x},
\end{eqnarray}
where $I(\vecx{x})$ is the image intensity at pixel $\vecx{x}$, $H$ is the Heaviside function, $K_\sigma$ is a gaussian kernel defined as:
\begin{equation}
	\label{eq:Ksigma_li}
	K_\sigma(\vecx{u}) = \frac{1}{(2\pi)^{n/2}\sigma^n}e^{-\|\vecx{u}\|^2/2\sigma^2},
\end{equation}
with a scale parameter $\sigma > 0$. $f_1$ and $f_2$ are two functions centered at pixel $\vecx{x}$ and defined as:
\begin{equation}
	\label{eq:f1_li}
	f_1(\vecx{x}) = \frac{K_\sigma \ast (H(\phi(\vecx{x}))I(\vecx{x}))}{K_\sigma \ast H(\phi(\vecx{x}))},
\end{equation}
\begin{equation}
	\label{eq:f2_li}
	f_2(\vecx{x}) = \frac{K_\sigma \ast ((1 - H(\phi(\vecx{x})))I(\vecx{x}))}{K_\sigma \ast (1- H(\phi(\vecx{x})))}.
\end{equation}

The two first integrals of \refeq{eq:NRJ_li} correspond to data attached term. The third integral is a regularization term that minimizes the curve length. The last integral is a regularization term that forces the \ls~to keep signed distance properties over the evolution process.

\paragraph{Evolution equation}
\begin{eqnarray}
	\label{eq:evol_li}
	\nonumber \frac{\partial \phi}{\partial t}(\vecx{x}) = & \delta(\phi(\vecx{x})) \left( \lambda_1 \int K_\sigma(\vecx{x} - \vecx{y}) | I(\vecx{y}) - f_1(\vecx{x}) |^2 \, d\vecx{y} + \lambda_2 \int \int K_\sigma(\vecx{x} - \vecx{y}) | I(\vecx{y}) - f_2(\vecx{x}) |^2 \, d\vecx{y} \right) \\ 
	& + \nu \delta(\phi(\vecx{x})) \text{div}\left(\frac{\nabla \phi(\vecx{x})}{\|\nabla \phi(\vecx{x})\|}\right) + \mu \left( \nabla^2 \phi(\vecx{x}) - \text{div}\left(\frac{\nabla \phi(\vecx{x})}{\|\nabla \phi(\vecx{x})\|}\right) \right),
\end{eqnarray}

\paragraph{Properties}

\begin{itemize}
	\item Because of the localization effects introduced by $f_1$, $f_2$ and $K_\sigma$, this algorithm is able to segment inhomogeneous objects.
	\item This algorithm segments the whole image.
\end{itemize}


\newpage
\section[Lankton]{Lankton \cite{Lankton2008}}
\label{sec:Lankton}

\paragraph{Energy criterion}
\begin{equation}
	\label{eq:NRJ_lankton}
	E(\phi) = \int_{\Omega_x} \delta(\phi(\vecx{x})) \int_{\Omega_y} B(\vecx{x},\vecx{y}) \cdot F(I(\vecx{y}), \phi(\vecx{y})) \, d\vecx{y}d\vecx{x} + \lambda \int_{\Omega_x} \delta(\phi(\vecx{x})) \| \nabla \phi(\vecx{x}) \|d\vecx{x},
\end{equation}
where $\delta$ is the Dirac function, $B$ is a ball of radius $r$ centered at point $\vecx{x}$ and defined as follow:
\begin{equation}
	\label{eq:B}
	B(\vecx{x},\vecx{y}) = 
	\begin{cases}
		1, & \|\vecx{x}-\vecx{y}\| \leq r \\
		0, & otherwise,
	\end{cases}
\end{equation}
and
\begin{equation}
	\label{eq:F_lankton}
	F(I(\vecx{y}),\phi(\vecx{y})) = 
	\begin{cases}
		H(\phi(\vecx{y}))(I(\vecx{y}) - v_\vecx{x})^2 + (1 - H(\phi\vecx{y})))(I(\vecx{y}) - u_\vecx{x})^2, & \text{Chan \& Vese feature,}\\
		(v_\vecx{x} - u_\vecx{x})^2, & \text{Yezzi feature,}
	\end{cases}
\end{equation}
where $H$ is the Heaviside function, $u_\vecx{x}$ and $v_\vecx{x}$ are two parameters updated at each iteration as follows:

\begin{equation}
	\label{eq:u_lankton}
	u_\vecx{x} = \frac{\int_{\Omega_y} B\vecx{x},\vecx{y}) \cdot (1 - H(\phi(\vecx{y}))) \cdot \vecx{y})\, d\vecx{y}}{\int_{\Omega_y} B(\vecx{x},\vecx{y}) \cdot (1 - H(\phi(\vecx{y}))) \, d\vecx{y}} 
\end{equation}
\begin{equation}
	\label{eq:v_lankton}
	v_\vecx{x} = \frac{\int_{\Omega_y} B(\vecx{x},\vecx{y}) \cdot H(\phi(\vecx{y})) \cdot I(\vecx{y})\, d\vecx{y}}{\int_{\Omega_y} B(\vecx{x},\vecx{y}) \cdot H(\phi(\vecx{y})) \, d\vecx{y}} 
\end{equation}

The first integral of \refeq{eq:NRJ_lankton} correspond to a data attached term and the second is a regularization term that acts on the evolving contour.

\paragraph{Evolution equation}
\begin{equation}
	\label{eq:evol_lankton}
	\frac{\partial \phi}{\partial t}(\vecx{x}) = \delta(\phi(\vecx{x})) \int_{\Omega_y} B(\vecx{x},\vecx{y}) \cdot \nabla_\phi F(I(\vecx{y}), \phi(\vecx{y})) \, d\vecx{y} + \lambda \delta(\phi(\vecx{x})) \text{div}\left(\frac{\nabla \phi(\vecx{x})}{\|\nabla \phi(\vecx{x})\|}\right),
\end{equation}
where
\begin{equation}
	\label{eq:gradF_lankton}
	\nabla_\phi F(I(\vecx{y}),\phi(\vecx{y})) = 
	\begin{cases}
		\delta(\phi(\vecx{y}))((I(\vecx{y}) - v_\vecx{x})^2 - (I(\vecx{y}) - u_\vecx{x})^2), & \text{Chan \& Vese feature,}\\
		\delta(\phi(\vecx{y})) \left( \frac{(I(\vecx{y}) - v_\vecx{x})^2}{A_v} - \frac{(I(\vecx{y}) - u_\vecx{x})^2}{A_u}\right), & \text{Yezzi feature,}
	\end{cases}
\end{equation}
where $A_u$ and $A_v$ are the area of the local interior and local exterior regions respectively given by
\begin{equation}
	\label{eq:Au_lankton}
	A_u = \int_{\Omega_y} B(\vecx{x},\vecx{y}) \cdot (1 - H(\phi(\vecx{x}))) \, d\vecx{y}
\end{equation}
\begin{equation}
	\label{eq:Av_lankton}
	A_v = \int_{\Omega_y} B(\vecx{x},\vecx{y}) \cdot H(\phi(\vecx{x})) \, d\vecx{y}
\end{equation}

\paragraph{Properties}

\begin{itemize}
	\item This algorithm is a region-based method.
	\item Its feature term is computed locally. This property allows the algorithm to segment non homogeneous objects. However this make the method sensitive to initialization.
\end{itemize}


\newpage
\section[Bernard]{Bernard \cite{Bernard2009a}}
\label{sec:Bernard}

\paragraph{Model}
~\par \vspace{0.3cm}

Let $\Omega$ be a bounded open subset of $\Rset^{d}$ and let $f:\Omega\mapsto\Rset$ be a given $d$-dimensional image. In the \bs~\ls~ formalism, the evolving interface $\Gamma\subset\Rset^{d}$ is represented as the zero level-set of an implicit function \LS~expressed as a linear combination of \bs~ basis functions
\begin{equation}
\label{eq:phi-bernard}
\phi({\vecx{x}})=\sum_{{\vecx{k}}\in{\Zset}^{d}}\,c[{\vecx{k}}]\,\beta^{n} \left(\frac{{\vecx{x}}}{h}-{\vecx{k}}\right).
\end{equation}

Here, $\beta^{n}(\cdot)$ is the uniform symmetric $d$-dimensional \bs~ of degree $n$. The knots of the \bs~ are located on a grid spanning $\Omega$, with a regular spacing. The coefficients of the \bs~ representation are gathered in $c[\vecx{k}]$. $h$ is a scale parameter which directly influence the degree of smoothing of the interface.

\paragraph{Energy criterion}
\begin{equation}
	\label{eq:NRJ-bernard}
	E(\phi) = \int_{\Omega} F(I(\vecx{x}),\phi(\vecx{x})) \, d\vecx{x},
\end{equation}
where 
\begin{equation}
	\label{eq:F_bernard}
	F(I(\vecx{x}),\phi(\vecx{x})) = H(\phi(\vecx{x}))(I(\vecx{x}) - v)^2 + (1 - H(\phi(\vecx{x})))(I(\vecx{x}) - u)^2,
\end{equation}
$H$ is the Heaviside function, $u$ and $v$ are two parameters updated at each iteration according to equation \refeq{eq:u_chanvese} and \refeq{eq:v_chanvese}.

\paragraph{Evolution equation}
~\par \vspace{0.3cm}

The minimization of the functional \refeq{eq:NRJ-bernard} can be done with respect to the \bs~coefficients $c[\vecx{k}]$. The derivatives with respect to each \bs~coefficient $c[\vecx{k}_{\textbf{0}}]$ may be expressed as
\begin{equation}
	\label{eq:dEdck-bernard}
	\frac{\partial E}{\partial c[\vecx{k}_{\vecx{0}}]} = \int_{\Omega}{\frac{\partial F(\vecx{x},\phi(\vecx{x}))}{\partial \phi(\vecx{x})} \cdot \beta^{n}\left( \frac{\vecx{x}}{h} - \vecx{k}_{\vecx{0}} \right) \, d\vecx{x}},
\end{equation}
with 
\begin{equation}
	\label{eq:gradF-bernard}
	\frac{\partial F(\vecx{x},\phi(\vecx{x}))}{\partial \phi(\vecx{x})} = \delta(\phi(\vecx{x})( (I(\vecx{x})-v)^2 - (I(\vecx{x}) - u)^2 ).
\end{equation}

The level-set evolution may then be computed through a gradient descent on the \bs~coefficients. The corresponding variation of the \bs~coefficients is given as:
\begin{equation}
	\label{eq:evol-bernard}
	\vecx{c}^{i+1} = \vecx{c}^{i} - \lambda\nabla_{c}E(\vecx{c}^{i}),
\end{equation}
where $\lambda$ is the iteration step and $\nabla_{c}$ correspond to the gradient of the energy relative to the \bs~coefficients given by \refeq{eq:dEdck-bernard}.

\paragraph{Properties}

\begin{itemize}
	\item This algorithm computes the \ls~evolution on the whole image. So new contours could emerge far from the initialization.
	\item This algorithm is a region-based method and tries to separate the image into two homogeneous region (according to their means value).
\end{itemize}


\newpage
\section[Shi]{Shi \cite{Shi2008}}
\label{sec:Shi}

\paragraph{Model}
~\par \vspace{0.3cm}

\hspace{0.7cm} This method is a fast algorithm that approximate \ls~based curve evolution. The implicit function is represented using a limited set of integers (-3, -1, 1, 3) to define the interior points, the interior points adjacent to the evolving curve, the exterior points adjacent to the evolving curve, the exterior points. Moreover the points adjacent to the evolving curve are gathered into two lists: $L_{in}$ and $L_{out}$.
\vspace{0.3cm}

The curve evolution process is then approximated in a two-cycle algorithm:
\begin{enumerate}
	\item during Na iterations, the curve evolve using a data attachment term $F_d$.
	\item then the curve is smoothed during Ns iterations; the regularization term $F_r$ is computed for each point of $L_{in}$ and $L_{out}$ using a gaussian filter of variance $\sigma$ and size Ng$\times$Ng.
\end{enumerate}

\paragraph{Evolution equation}
~\par \vspace{0.3cm}
In \cite{Shi2008}, several data attached term are given. In the platform, we choose to use the Chan \& Vese one given by:
\begin{equation}
	\label{eq:F_shi}
	F(I(\vecx{x}),\phi(\vecx{x})) = H(\phi(\vecx{x}))(I(\vecx{x}) - v)^2 + (1 - H(\phi(\vecx{x})))(I(\vecx{x}) - u)^2
\end{equation}
where $H$ is the Heaviside function, $u$ and $v$ are two parameters updated at each iteration according to equation \refeq{eq:u_chanvese} and \refeq{eq:v_chanvese}.

\paragraph{Properties}

\begin{itemize}
	\item This method is an approximation of \ls~based curve evolution. It is a very fast method and evolves only on the narrow-band.
\end{itemize}

% ----------------- %
% A priori de forme %
% ----------------- %

\newpage
\chapter{\Ls~with shape prior}
\label{chap:shape}

\section[Leventon \etal]{Leventon \etal \cite{Leventon2000}}
\label{sec:shape-leventon}

\paragraph{Curve representation}
~\par \vspace{0.3cm}

Each curve in the training dataset is embedded as the zero level set of a higher dimensional surface $\phi$, whose height is sampled at regular intervals (say $N^d$ samples, where $d$ is the number of dimensions). The embedding function chosen is the commonly used signed distance function, where each sample encodes the distance to the nearest point on the curve, with negative values inside the curve. Each such surface (distance map) can be considered a point in a high dimensional space ($\phi \in \Rset^{N^d}$). The training set $T$, consists of a set of surfaces $T = \{\phi_1, \cdots, \phi_n \}$. The goal is to build a shape model over this distribution of surfaces. Since a signed distance map is uniquely determined from the zero level set, each distance map has a large amount of redundancy. Furthermore, the collection of curves in the training set presumably has some dependence, as they are shapes of the same class of object, introducing more redundancy in the training set. The cloud of points corresponding to the training set is approximated to have a Gaussian distribution, where most of the dimensions of the Gaussian collapse, leaving the principal modes of shape variation.

The mean surface $\mu$, is computed by taking the mean of the signed distance functions, $\mu = \frac{1}{n} \sum_{i=1}^n \phi_i$. The variance in shape is computed using Principal Component Analysis (PCA). The mean shape $\mu$ is subtracted from each $\phi_i$ to create an mean-offset map $\hat{\phi}_i$. Each such map $\hat{\phi}_i$ is placed as a column vector in an $N^d \times n$-dimensional matrix $M$. Using Singular Value Decomposition (SVD), the covariance matrix $\frac{1}{n}MM^T$ is decomposed as:
\begin{equation}
  \label{eq:decomp_leventon}
  U\Sigma U^T = \frac{1}{n}MM^T
\end{equation}
where $U$ is a matrix whose column vectors represent the set of orthogonal modes of shape variation and $\Sigma$ is a diagonal matrix of corresponding singular values. An estimate of a novel shape $\phi$ of the same class of object can be represented by $k$ principal components in a $k$-dimensional vector of coefficients $\alpha = U_k^T(\phi-\mu)$, where $U_k$ is a matrix consisting of the $k$ first columns of $U$ that is used to project a surface into the eigen-space. Given the coefficients $\alpha$, an estimate of the shape $\phi$, namely $\tilde{\phi}$, is reconstructed from $U_k$ and $\mu$:
\begin{equation}
  \label{eq:recomstr_leventon}
  \tilde{\phi} = \alpha U_k + \mu.
\end{equation}
Under the assumption of a Gaussian distribution of shape represented by $\alpha$, the probability of a certain curve can be computed as:
\begin{equation}
  \label{eq:prob_leventon}
  P(\alpha) = \frac{1}{\sqrt{(2\pi)^k|\Sigma_k|}}exp \left(-\frac{1}{2}\alpha^T\Sigma_k^{-1}\alpha \right),
\end{equation}
where $\Sigma_k$ contents the $k$ first rows and columns of $\Sigma$.


\paragraph{Evolution equation}
\begin{equation}
  \label{eq:dphidt-leventon}
  \phi(t+1) = \phi(t) + \lambda_1[g(c + \kappa) \|\nabla\phi(t)\| + \nabla\phi(t)\cdot\nabla g] + \lambda_2(\phi^\ast(t) - \phi(t)),
\end{equation}
where $c$ is an image-dependent balloon force added to force the contour to flow outward, $\lambda_1$ is a parameter defining the update step size, $\lambda_2 \in [0;1]$ is the linear coefficient that determines how much to trust the maximum a posteriori estimate and $\phi^\ast$ is the estimate of the final surface:
\begin{equation}
  \label{eq:phi*-leventon}
  \phi_{MAP}^\ast = \underset{\phi^\ast}{argmax}(P(\phi^\ast \; | \; \phi, \nabla I)),
\end{equation}
which is completely determined by the shape $\alpha$ and the pose $p$:
\begin{equation}
  \label{eq:shape-pose-leventon}
  \langle\alpha_{MAP}, p_{MAP}\rangle = \underset{\phi^\ast}{argmax}(P(\alpha, p \; | \; \phi, \nabla I)).
\end{equation}
To compute the maximum a posteriori final curve, the terms from eq. \refeq{eq:shape-pose-leventon} are expanded using Bayes' Rule (proof in Appendix \ref{par:Proof-MAP-leventon}):
\begin{equation}
  \label{eq:MAP-leventon}
  P(\alpha, p \; | \; \phi, \nabla I) = \frac{P(\phi \; | \; \alpha,p)P(\nabla I \; | \; \alpha, p, \phi)P(\alpha)P(p)}{P(\phi,\nabla I)}.
\end{equation}


\paragraph{Proof of equation \refeq{eq:MAP-leventon}}
\label{par:Proof-MAP-leventon}
~\par \vspace{0.3cm}
\emph{Note 1:} $P(A,B) = P(A \cap B)$

\emph{Note 2:} Bayes' rule
\begin{eqnarray}
  \label{eq:sec-bayes}
  & P(A \cap B) = P(B \; | \; A)P(A) = P(A \; | \; B)P(B) \label{eq:sec-bayes1}\\
  & \Rightarrow P(A \; | \; B) = \dfrac{P(B \; | \; A)P(A)}{P(B)} \label{eq:sec-bayes2}.
\end{eqnarray}

Using Bayes' rule \refeq{eq:sec-bayes2} we can write:
\begin{equation}
  \label{eq:sec-MAP-leventon1}
  P(\alpha, p \; | \; \phi, \nabla I) = \frac{P(\phi,\nabla I \; | \; \alpha, p)P(\alpha,p)}{P(\phi, \nabla I)}
\end{equation}
Then using Bayes' rule \refeq{eq:sec-bayes1}, the first term of the numerator can be modified as follows:
\begin{align}
  \label{eq:sec-MAP-leventon2}
  \nonumber P(\phi,\nabla I \; | \; \alpha, p) & = \dfrac{P(\phi,\nabla I,\alpha, p)}{P(\alpha,p)} = \dfrac{P(\nabla I \cap \phi \cap \alpha \cap p)}{P(\alpha,p)} \\
  \nonumber & = \dfrac{P(\nabla I \; | \; \phi,\alpha,p)P(\phi,\alpha,p)}{P(\alpha,p)} = \dfrac{P(\nabla I \; | \; \phi,\alpha,p)P(\phi \; | \; \alpha, p)P(\alpha,p)}{P(\alpha,p)} \\
  & = P(\nabla I \; | \; \phi,\alpha,p)P(\phi \; | \; \alpha, p)
\end{align}

Replacing equation \refeq{eq:sec-MAP-leventon2} into equation \refeq{eq:sec-MAP-leventon1} and assuming that shape is independent from pose yields to:
\begin{equation}
  \label{eq:sec-MAP-leventon3}
  P(\alpha, p \; | \; \phi, \nabla I) = \dfrac{P(\phi \; | \; \alpha,p)P(\nabla I \; | \; \alpha, p, \phi)P(\alpha)P(p)}{P(\phi,\nabla I)}.
\end{equation}


\newpage
\section[Chen \etal]{Chen \etal \cite{Chen2002}}
\label{sec:shape-chen}

\paragraph{Energy criterion}
~\par \vspace{0.3cm}
Let $C^\ast$ be a curve, called the shape prior, representing the expected shape of the boundary.

\begin{equation}
  \label{eq:NRJ-chen}
  E(C, \mu, R, T) = \int_0^1 \left[ g(\| \nabla I(C(p))\|) + \frac{\lambda}{2}d^2(\mu RC(p) +T) \right] \|C'(p)\|dp,
\end{equation}
where $g$ is defined as in equation \refeq{eq:g_caselles}, $I(\cdot)$ corresponds to the image intensity, $C$ is the parametric curve, $\lambda$ is a parameter that weights the influence of the shape prior term and $d(x, y) = d(C^\ast, (x, y))$ is the distance of the point $(x, y)$ from $C^\ast$.

\vspace{0.3cm}
\emph{Note}: Two curves $C_1$ and $C_2$ have the same shape, if there exist a scale $\mu$, a rotation matrix $R$ (rotation by an angle $\theta$ ) and a translation vector $T$ such that $C_1$ coincides with $C_2^{new} = \mu RC_2 + T$.

\vspace{0.3cm}
In the \ls~formalism, equation \refeq{eq:NRJ-chen} can be rewriten as follows:
\begin{equation}
  \label{eq:NRJ-chen2}
  E(C, \mu, R, T) = \int_\Omega \delta(\phi(\vecx{x})) \left[ g(\| \nabla I(\vecx{x}))\|) + \frac{\lambda}{2}d^2(\mu R\vecx{x} +T) \right] \| \nabla \phi(\vecx{x}) \| d\vecx{x},
\end{equation}
where $\delta$ is the dirac function and $\Omega$ the definition domain.

\paragraph{Evolution equation}
~\par \vspace{0.3cm}
The gradient/variation descent for the energy of equation \refeq{eq:NRJ-chen2} is performed by the following system:
\begin{equation}
  \label{eq:dphidt-chen}
  \frac{\partial \phi}{\partial t} = \delta(\phi) \text{div}\left[\left(g + \frac{\lambda}{2}d^2\right)\frac{\nabla \phi}{\| \nabla \phi \|}\right], \, x \in \Omega, \, t > 0,
\end{equation}
\begin{equation}
  \label{eq:dphidn-chen}
  \frac{\partial \phi}{\partial \vec{N}} = 0, \, x \in \delta\Omega, \, t > 0, \, \phi(x, 0) = \phi_0(x),
\end{equation}
\begin{equation}
  \label{eq:dmudt-chen}
  \frac{\partial \mu}{\partial t} = -\lambda \int_\Omega \delta(\phi(\vecx{x})) d \nabla d \cdot (R\vecx{x}) \| \nabla \phi(\vecx{x}) \| d\vecx{x}, \, t>0, \, \mu(0) = \mu_0,
\end{equation}
\begin{equation}
  \label{eq:dthetadt-chen}
  \frac{\partial \theta}{\partial t} = -\lambda \mu \int_\Omega \delta(\phi(\vecx{x})) d \nabla d \cdot \left( \frac{dR}{d\theta} \vecx{x} \right) \| \nabla \phi(\vecx{x}) \| d\vecx{x}, \, t>0, \, \theta(0) = \theta_0,
\end{equation}
\begin{equation}
  \label{eq:dTdt-chen}
  \frac{\partial T}{\partial t} = -\lambda \int_\Omega \delta(\phi(\vecx{x})) d \nabla d \| \nabla \phi(\vecx{x}) \| d\vecx{x}, \, t>0, \, T(0) = T_0,
\end{equation}
where the function $d$ is evaluated at $\mu RC(p) + T$.


\paragraph{Curve representation}
~\par \vspace{0.3cm}
A set of training images which contain example curves with similar shape is used to extract an average of these curves. The training set contains $n$ given curve $C_1, \cdots, C_n$ with similar shape, but different size, orientation and translation. Let $A_1$ and $A_2$ denote the interior regions of the curves $C_1$ and $C_2$ as subsets of $\Rset^2$. Define
\begin{equation}
  \label{eq:a-chen}
  a(C_1, C_2) = area \, of (A_1 \cup A_2 - A_1 \cap A_2).
\end{equation}
$a(C_1, C_2)$ is used to measure the similarity of the shapes of $C_1$ and $C_2$. To compare the shapes of the $n$ contours, $C_1$ is fixed, and $C_j$ $(j=2, \cdots, n)$ are realigned to $C_1$ by finding a scale $\mu_j$, a rotation matrix $R_j$ and a translation vector $T_j$ such that the area $a(C_1, C_j^{new})$ is minimized, where $C_j^{new} = \mu_j R_j C_j + T_j$. After the realignment of these curves, the average shape is computed:
\begin{equation}
  \label{eq:C*-chen}
  C^* = \frac{C_1 + \sum_{j=2}^n {C_j^{new}}} {n}.
\end{equation}



\newpage
\section[Rousson and Paragios]{Rousson and Paragios \cite{Rousson2002}}
\label{sec:shape-rousson}

\paragraph{Curve representation}
~\par \vspace{0.3cm}

A shape model is generated that accounts for global statistical and local variations. In order to do that, a stochastic framework with two unknown variables:
\begin{itemize}
 \item the shape image, $\phi_M (x, y)$,
 \item the local degrees (variability) of shape deformations $\sigma_M (x,y)$,
\end{itemize}
is considered, where each grid location can be described in the shape model using a Gaussian density function

\begin{equation}
  \label{eq:DensityFunctions-rousson}
  p_{x,y}^M(\phi) = \frac{1}{\sqrt{2\pi}\sigma_M(x,y)} \exp\left(-\frac{(\phi-\phi_M(x,y))^2}{2\sigma_M^2(x,y)} \right).
\end{equation}

The mean of this probability density function corresponds to the \ls~function, while the variance refers to the variation of the aligned samples in this location. On top of these assumptions, a constraint is imposed that mean values of the shape model
refer to a signed distance function (level set representation).

Thus given $N$ aligned training samples (\ls~representations) where $\hat{\phi_i}$ is the aligned transformation of $\phi_i$, a variational framework can be constructed for the estimation of the best shape by seeking for the maximum likelihood of the local densities with respect to ($\phi_M$, $\sigma_M$):
\begin{align}
  \label{eq:NRJAlign-rousson}
  \nonumber E(\phi_M,\sigma_M) = & (1-\alpha)\int_\Omega \left( \left(\frac{d}{dx}\sigma_M(x,y)\right)^2 + \left(\frac{d}{dy}\sigma_M(x,y)\right)^2 \right) \, dxdy \\
  & + \alpha \int_\Omega \sum_{i=1}^N{ \left(\log[\sigma_M(x,y)] + \frac{(\hat{\phi_i}(x,y) - \phi_M(x,y))^2}{2\sigma_M^2(x,y)}\right) } \, dxdy
\end{align}

\paragraph{Energy criterion}
~\par \vspace{0.3cm}

Given the current \ls~representation $\phi$, one can assume that there is an ideal transformation $A = (Ax, Ay)$ between the shape prior and the evolving representation. During the model construction, the shape model was considered as having some local degrees of variability. In that case the ideal transformation will map each value of current representation at the most probable value on the model:
\begin{align}
  \label{eq:Transform-rousson}
  \begin{cases}
    \nonumber & (x,y) \rightarrow A(x,y) \\
    max_{x,y}\left(p_{A(x,y)}^M(s\phi(x,y))\right), \, \forall (x,y) & H(\phi(x,y)) \geq 0.
  \end{cases}
\end{align}
The most probable transformation is the one ontained through the maximum likelihood for all pixels. Under the assumption that densities are independent across pixels, the minimization of the log-function of the maximum likelihood can be considered as global optimization criterion. This criterion refers to two set of unknown variables. The linear transformation $A$, and the \ls~function $\phi$:
\begin{eqnarray}
  \label{eq:NRJ-rousson}
  \nonumber E(\phi,A) = & -\int_\Omega H(\phi(\vecx{x})) & \log \left( p_{A(x,y)}^M(s\phi(\vecx{x})) \right) \, d\vecx{x} \\
  = & \int_\Omega H(\phi(\vecx{x})) & \left( \log(\sigma_M(A(\vecx{x}))) + \dfrac{(s\phi(\vecx{x}) - \phi_M(A(\vecx{x})))^2}{2\sigma_M^2(A(\vecx{x}))}\right) \, d\vecx{x}
\end{eqnarray}


\paragraph{Evolution equation}
\begin{equation}
  \label{eq:dphidt-rousson}
  \frac{\partial}{\partial t}\phi(\vecx{x}) = -sH(\phi(\vecx{x}))\left(\frac{s\phi(\vecx{x}) - \phi_M(A(\vecx{x}))}{\sigma_M^2(A(\vecx{x}))} \right) - \delta(\phi(\vecx{x}))\left(\log(\sigma_M(A(\vecx{x}))) + \frac{(s\phi(\vecx{x}) - \phi_M(A(\vecx{x})))^2}{2\sigma_M^2(A(\vecx{x}))} \right)
\end{equation}



\newpage
\section[Cremers \etal]{Cremers \etal 2003 \cite{Cremers2003b}}
\label{sec:shape-cremers2003b}

\paragraph{Energy criterion}
\begin{equation}
  \label{eq:NRJ_cremers2003b}
  E(\phi) = \int_\Omega F(I(\vecx{x}), \phi(\vecx{x})) \, d\vecx{x} + \nu \int_\Omega \| \nabla H(\phi(\vecx{x})) \| d\vecx{x} + \alpha E_{shape}(\phi),
\end{equation}
where $F(I(\vecx{x}), \phi(\vecx{x}))$ is defined as in equation \refeq{eq:F_chanvese} and
\begin{equation}
  \label{eq:Eshape_cremers2003b}
  E_{shape}(\phi,L) = \int_\Omega (\phi(\vecx{x}) - \phi_0(\vecx{x}))^2 (L + 1)^2 \, d\vecx{x} \; + \, \int_\Omega \lambda^2 (L - 1)^2 \, d\vecx{x} \; + \, \gamma \int_\Omega \| \nabla H(L) \| \, d\vecx{x},
\end{equation}
where $\phi_0$ is the \ls~function embedding a given training shape (or the mean of a set of training shapes) and $L$ is a labeling function ($L \, : \, \Omega \, \mapsto \, \Rset$) that models a selective shape prior  and indicates the areas of the image plane in which a given prior should be enforced. This labeling function is to take on the values $+1$ and $-1$ depending on whether the prior should be enforced or not.

\emph{Note:} The energy defined in equation \refeq{eq:Eshape_cremers2003b} does not take into account the pose of the object. This can be done by introducing  a set of pose parameters associated with a given prior (\cite{Chen2002, Rousson2002, Cremers2006}). The corresponding shape energy 
\begin{align}
  \label{eq:Eshape_pose_cremers2003b}
  \nonumber E_{shape}(\phi,L,s,R_\theta,h) = & \int_\Omega (\phi(\vecx{x}) - \frac{1}{s}\phi_0(s R_\theta\vecx{x} + h))^2 (L + 1)^2 \, d\vecx{x} \\
	& + \, \int_\Omega \lambda^2 (L - 1)^2 \, d\vecx{x} \;  + \, \gamma \int_\Omega \| \nabla H(L) \| \, d\vecx{x},
\end{align}
is simultaneously optimized with respect to the segmenting \ls~function $\phi$, the labeling function $L$ and transformation parameters, which account for translation $h$, rotation by an angle $\theta$ and scaling $s$ of the template. The division by $s$ guarantees that the resulting shape remains a distance function.


\paragraph{Evolution equation}
~\par \vspace{0.3cm}
For fixed labeling $L$ and pose parameters $p$, local optimization of the \ls~function $\phi$ leads to
\begin{equation}
  \label{eq:evol_cremers2003b}
  \frac{\partial \phi}{\partial t}(\vecx{x}) = \delta(\phi(\vecx{x})) \left( \nabla_\phi F(I(\vecx{x}), \phi(\vecx{x})) + \nu \text{div}\left(\frac{\nabla \phi(\vecx{x})}{\|\nabla \phi(\vecx{x})\|}\right) \right) - 2\alpha (L + 1)^2(\phi - \phi_0),
\end{equation}
where $\nabla_\phi F(I(\vecx{x}),\phi(\vecx{x}))$ is defined as in equation\refeq{eq:gradF_chanvese}.

For fixed labeling $L$ and \ls~function $\phi$, local optimization of the pose parameters $p$ can be implemented by gradient descent. With $\vecx{g} = s R_\theta \vecx{x} + h$, the evolution equations for translation $h$ , rotation $\theta$ and scaling $s$ associated with the shape model $\phi_0$, are given by (see also \cite{Cremers2006}):
\begin{equation}
  \label{eq:h_evol_cremers2003b}
  \frac{\partial h}{\partial t} =  2\int_\Omega \frac{(\phi(\vecx{x}) - \frac{1}{s}\phi_0(\vecx{g}))}{s} (L + 1)^2 \nabla\phi_0(\vecx{g}) \, d\vecx{x},
\end{equation}
\begin{equation}
  \label{eq:theta_evol_cremers2003b}
  \frac{\partial \theta}{\partial t} =  2\int_\Omega (\phi(\vecx{x}) - \frac{1}{s}\phi_0(\vecx{g})) (L + 1)^2 \nabla\phi_0(\vecx{g})^T \frac{\partial R_\theta}{\partial \theta} \vecx{x} \, d\vecx{x},
\end{equation}
\begin{equation}
  \label{eq:s_evol_cremers2003b}
  \frac{\partial s}{\partial t} =  2\int_\Omega \frac{(\phi(\vecx{x}) - \frac{1}{s}\phi_0(\vecx{g}))}{s} (L + 1)^2 [\nabla\phi_0(\vecx{g})^T R_\theta \vecx{x} - \frac{1}{s}\phi_0(\vecx{g})] \, d\vecx{x},
\end{equation}


\paragraph{Labeling function}
~\par \vspace{0.3cm}
According to the proposed cost functional, the labeling will evolve in an unsupervised manner driven by two criteria:
\begin{enumerate}
 \item The labeling should enforce the shape prior in those areas of the image where the \ls~function is similar to the prior. In particular, for fixed $\phi$, minimizing the first two terms in equation \refeq{eq:Eshape_cremers2003b} will generate the following qualitative behavior of the labeling:
 \begin{itemize}
   \item $L \rightarrow +1$,  if $|\phi - \phi_0 | < \lambda$,
   \item $L \rightarrow -1$, if $|\phi - \phi_0 | > \lambda$;
 \end{itemize}
 \item The boundary separating regions with shape prior from regions without shape prior should have minimal length. This regularizing constraint on the zero crossing of the labeling function - given by the last term in equation \refeq{eq:Eshape_cremers2003b} - induces a topological ``compactness`` of the regions with and without shape prior.
\end{enumerate}

For fixed $\phi$, the gradient descent equation for the labeling function is given by:
\begin{equation}
  \label{eq:evol_label_cremers2003b}
  \frac{\partial L}{\partial t} = -\frac{\partial E}{\partial L} = \alpha \left( 2\lambda^2(1-L) \, - \, 2(\phi - \phi_0)^2 (1+L) \, + \, \gamma \delta(L) \text{div} \left( \frac{\nabla L}{\| \nabla L \|} \right) \right).
\end{equation}
The first two terms drive the labeling toward $-1$ or $+1$, depending on whether $|\phi - \phi_0|$ is larger or smaller than $\lambda$. And the last term in \refeq{eq:evol_label_cremers2003b} minimizes the length of the zero-crossing of $L$, thereby enforcing decision regions with minimal boundary.



\newpage
\section[Chan \etal]{Chan \etal 2005 \cite{Chan2005}}
\label{sec:shape-chan2005}

\paragraph{Shape representation}
~\par \vspace{0.3cm}
Given an object $\Omega \in \Rset^2$, which is assumed to be closed and bounded, there is a unique solution to the following equation:
\begin{align}
 \label{eq:chan2005-shape}
  \| \nabla \phi \| = 1, \\
  \begin{cases}
     \phi(x,y) > 0, & (x,y) \in \Omega \backslash \partial \Omega \\
     \phi(x,y) = 0, & (x,y) \in \partial \Omega \\
     \phi(x,y) < 0, & (x,y) \in \Rset^2 \backslash \Omega \\
  \end{cases}
\end{align}
Hence, any object in the plane corresponds a unique signed distance function, and vice versa.

As a shape is invariant to translation, rotation and scaling, we may define an equivalent relation in the collection of objects in the plane. Any two objects are said to be equivalent if they have the same shape. Their signed distance functions are related. For example, let $\Omega_1$ and $\Omega_2$ be two objects with the same shape, and $\phi_1$ and $\phi_2$ be the signed distance functions respectively, then there exists a four-tuple $p = (a, b, r, \theta)$ such that:
\begin{equation}
 \label{eq:chan2005-shape2}
  \phi_2(x,y) = r \phi_1 \left( \frac{(x-a)\cos\theta + (y-b)\sin\theta}{r}, \frac{-(x-a)\sin\theta + (y-b)\cos\theta}{r} \right),
\end{equation}
where $(a, b)$ represents the center, $r$ the scaling factor and $\theta$ the angle of rotation. In this way, given any object,
consequently a signed distance function, we may get the representation of other objects in the equivalent class by
choosing the four-tuple $p = (a, b, r, \theta)$.


\paragraph{Energy criterion}
\begin{equation}
  \label{eq:NRJ_chan2005}
  E(\phi,L,\psi) = E_{CV}(\phi) + E_{shape}(\phi,L,\psi) + E_{\psi}(\psi),
\end{equation}
where
\begin{equation}
  \label{eq:Ecv_chan2005}
  E_{CV}(\phi) = \int_\Omega F(I(\vecx{x}), \phi(\vecx{x})) \, d\vecx{x},
\end{equation}
$I(\cdot)$ corresponds to the image intensity, $F(I(\vecx{x}), \phi(\vecx{x}))$ is defined as in equation \refeq{eq:F_chanvese},
\begin{align}
  \label{eq:Eshape_chan2005}
  \nonumber E_{shape}(\phi,L,\psi) = \lambda & \int_\Omega (H(\phi(\vecx{x}))H(L(\vecx{x})) - H(\psi(\vecx{x})))^2 \, d\vecx{x} \\
	& + \mu_1 \int_\Omega (1- H(L(\vecx{x}))) d\vecx{x} + \mu_2 \int_\Omega \| \nabla H(L(\vecx{x})) \| d\vecx{x},
\end{align}
where $\phi$ is the segmentation function, $\psi$ is the shape function and $L$ is equivalent to the labelling function in Cremers \etal \cite{Cremers2003b}. The prior shape will be compared with the region where both the level set function $\phi$ for segmentation and the labelling function $L$ are positive. The second term of equation \refeq{eq:Eshape_chan2005} encourages the area of the region $ \{ (x, y) \in \Omega : L(x, y) > 0 \}$, and the last one smoothes the boundary where $L$ separates the domain $\Omega$. $\lambda$, $\mu_1$, $\mu_2$ are non negative parameters.

However, it is elusive to choose an appropriate $\mu_1$. Too large $\mu_1$ will impair the action of the labelling function because the region $\{L > 0\}$ will contain other objects besides the desirable object, and on the other hand, if it is so small, $L$ could be stable at a state that the region $\{L > 0\}$ could be smaller than what it should be.

To overcome this dilemma, the last term of equation \refeq{eq:NRJ_chan2005} is introduced as:
\begin{equation}
  \label{eq:Epsi_chan2005}
  E_{\psi}(\psi) = \nu \int_\Omega F(I(\vecx{x}), \psi(\vecx{x})) \, d\vecx{x},
\end{equation}
where $F$ is defined as in equation \refeq{eq:F_chanvese} and $\nu$ is non negative parameters.

When the ideally segmentation for the goal object is obtained, the reference shape function $\psi$ should also segment the object from the image, so this term will be small.
\vspace{0.3cm}

\emph{Remark 1}: In the functional \refeq{eq:NRJ_chan2005}, the length term in $E_{CV}$ is omitted. It is partially because that the prior shape may control the smoothness of $\phi$ to some extend. However, the length term can be included in the above functional if it is necessary.

\emph{Remark 2}: With the term $E_\psi$ \refeq{eq:Epsi_chan2005}, the parameters $\mu_1$ and $\mu_2$ can be fixed (in \cite{Chan2005}, they both are fixed to 0.2).


\paragraph{Evolution equation}
\begin{equation}
  \label{eq:evol_chan2005}
  \frac{\partial \phi}{\partial t}(\vecx{x}) = -\delta(\phi(\vecx{x})) \left([(I(\vecx{x}) - v)^2 - (I(\vecx{x}) - u)^2] + 2\lambda H(L(\vecx{x})) [H(\phi(\vecx{x}))H(L(\vecx{x})) - H(\psi(\vecx{x}))] \right),
\end{equation}
where
\begin{equation}
  \label{eq:evol_chan2005_v}
  v = \frac{\int_\Omega I(\vecx{x}) \left(H(\phi(\vecx{x})) + \nu H(\psi(\vecx{x})) \right) \, d\vecx{x}}{\int_\Omega H(\phi(\vecx{x})) + \nu H(\psi(\vecx{x})) \, d\vecx{x}},
\end{equation}

\begin{equation}
  \label{eq:evol_chan2005_u}
  u = \frac{\int_\Omega I(\vecx{x}) \left( (1-H(\phi(\vecx{x}))) + \nu (1-H(\psi(\vecx{x}))) \right) \, d\vecx{x}}{\int_\Omega (1-H(\phi(\vecx{x}))) + \nu (1-H(\psi(\vecx{x}))) \, d\vecx{x}}.
\end{equation}

\begin{equation}
  \label{eq:evol_chan2005_L}
   \frac{\partial L}{\partial t}(\vecx{x}) = -\lambda H(\phi(\vecx{x}))(1-2H(\psi(\vecx{x}))) \| \nabla L(\vecx{x}) \| + \mu_1 \| \nabla L(\vecx{x}) \| + \mu_2 \| \nabla L(\vecx{x}) \| \nabla \left( \frac{\nabla L(\vecx{x})}{\| \nabla L(\vecx{x}) \|} \right).
\end{equation}

\begin{equation}
  \label{eq:evol_chan2005_a}
   \frac{\partial a}{\partial t} = \int_\Omega G(\phi, L, \psi) \left( \psi_{0x}(x^*, y^*) \sin \theta + \psi_{0y}(x^*,y^*) \cos \theta \right) \delta(\psi) \, d\vecx{x}d\vecx{y},
\end{equation}
\begin{equation}
  \label{eq:evol_chan2005_b}
   \frac{\partial b}{\partial t} = \int_\Omega G(\phi, L, \psi) \left( \psi_{0x}(x^*, y^*) \cos \theta + \psi_{0y}(x^*,y^*) \sin \theta \right) \delta(\psi) \, d\vecx{x}d\vecx{y},
\end{equation}
\begin{equation}
  \label{eq:evol_chan2005_r}
   \frac{\partial r}{\partial t} = \int_\Omega G(\phi, L, \psi) \left( -\psi_0(x^*, y^*) + \psi_{0x}(x^*, y^*) x^* + \psi_{0y}(x^*,y^*) y^* \right) \delta(\psi) \, d\vecx{x}d\vecx{y},
\end{equation}
\begin{equation}
  \label{eq:evol_chan2005_theta}
   \frac{\partial \theta}{\partial t} = \int_\Omega G(\phi, L, \psi) \left( -r \psi_{0x}(x^*, y^*) x^* + r \psi_{0y}(x^*,y^*) y^* \right) \delta(\psi) \, d\vecx{x}d\vecx{y},
\end{equation}
where
\begin{equation}
  \label{eq:evol_chan2005_G}
   G(\phi, L, \psi) = 2\lambda (H(\psi) - H(\phi)H(L)) + \nu \left( (I - v)^2 - (I - u)^2 \right),
\end{equation}
$x^* = \dfrac{(x-a) \cos \theta + (y - b) \sin \theta}{r}$, $y^* = \dfrac{-(x-a) \sin \theta + (y - b) \cos \theta}{r}$, $\psi_{0x} = \dfrac{\partial \psi_0}{\partial x}$, $\psi_{0y} = \dfrac{\partial \psi_0}{\partial y}$.



\newpage
\section[Foulonneau \etal]{Foulonneau \etal 2006 \cite{Foulonneau2006}}
\label{sec:shape-foulonneau}

\paragraph{Shape representation}
~\par \vspace{0.3cm}
Denoting by $\Omega_{in}$ the inside region of a shape, the regular or geometric moments of its characteristic function (which is binary) are defined as:
\begin{equation}
  \label{eq:moment_geometrique}
   M_{u,v} = \int\int_{\Omega_{in}}x^u y^v dxdy,
\end{equation}

where $(u, v) \in \Nset^2$ and $(u+v)$ is called the order of the moment. Any shape, discretized on a sufficiently fine grid, may be reconstructed from its infinite set of moments. Hence, when computed from the characteristic function, moments naturally provide region-based shape descriptors. However, a more tractable representation for reconstruction purposes is obtained by using an orthogonal basis, such as Legendre polynomials:
\begin{equation}
  \label{eq:moment_legendre}
   \lambda_{p,q} = C_{pq}\int\int_{\Omega_{in}} P_p(x) P_q(y) dxdy, \,\,\,\,\,\,\, (x,y) \in [-1,1]\times[-1,1],
\end{equation}
where the normalizing constant is: $C_{pq} = (2p+1)(2q+1)/4$ and
\begin{equation}
  \label{eq:poly_legendre}
   P_p(x) = \sum_{k=0}^p {a_{pk}x^k} = \frac{1}{2^p p!}\frac{d^p}{dx^p}(x^2-1)^p, \,\,\,\,\,\,\, x \in [-1,1],
\end{equation}

In practice, this representation is limited to a finite order $N$ and the shape descriptor is defined as $\vecx{\lambda} = \left\lbrace \lambda_{p,q}, p+q \leq N \right\rbrace$.

\emph{Note:} There is a linear relationship between Legendre moments and regular moments:
\begin{equation}
  \label{eq:rel_Legendre_regular}
   \lambda_{p,q} = C_{pq} \sum_{u=0}^p{\sum_{v=0}^q{ a_{pu}a_{qv}M_{u,v} }}
\end{equation}

Also, note that this description can take into account arbitrary shape topologies.


\paragraph{Energy criterion}
~\par \vspace{0.3cm}
 The shape prior is defined as a distance $d$ between the evolving curve $\Gamma$, and the reference in terms of shape descriptors:
\begin{equation}
  \label{eq:NRJ_foulonneau1}
  E_{prior}(\Omega_{in}(t)) = d(\lambda(\Omega_{in}),\lambda^{ref}),
\end{equation}
where $\Omega_{in}$ is the inside region of $\Gamma$ and $\lambda^{ref}$ is the set of moments of the reference object.

In the following, the case where the shape prior is a quadratic distance is developped:
\begin{equation}
  \label{eq:NRJ_foulonneau2}
  E_{prior}(\Omega_{in}(t)) = \sum_{p,q}^{p+q \leq N} {\left( \lambda_{p,q}(\Omega_{in}(t)) - \lambda_{p,q}^{ref} \right)^2}
\end{equation}

The simple case where the descriptor is invariant with regard to translation and scaling is considered, that is $\lambda$ and $\lambda^{ref}$ are computed from normalized central moments (see eq.(6) in \cite{Foulonneau2006}), i.e.,

\begin{equation}
  \label{eq:NRJ_moment_foulonneau}
  \lambda_{p,q} = C_{pq} \sum_{u=0}^p{ \sum_{v=0}^q{ a_{pu}a_{qv}\eta_{u,v} } }
\end{equation}


\paragraph{Evolution equation}
~\par \vspace{-0.5cm}
\subparagraph{General Framework}
\label{sspar:GenFW_foulonneau}
Since the set of regular open domains in $\Rset^2$ does not have a structure of vector space, classical gradient descent is impracticable for a region-based functional, $E(\Omega(t)$. Instead, we can consider that $\Omega$ evolves in a velocity vector field, $\vecx{V}$, and calculate the variations of in the direction $\vecx{V}$. To this end, we use the notion of Eulerian derivative \cite{Aubert2003}.

\textbf{Theorem:} \emph{The Eulerian derivative of the functional $E(\Omega(t) = \int \int_\Omega(t) k(\vecx{x},t) d\vecx{x}$ in the direction $\vecx{V}$ is given by:}
\begin{equation}
  \label{eq:dNRJ-aubert}
  \delta(E(\Omega(t))) = \int\int_{\Omega(t)}{\frac{\partial k}{\partial t}(\vecx{x},t) d\vecx{x}} - \int_{\Gamma(t)}{k(\vecx{x},t) \langle\vecx{V}.\vecx{N}\rangle ds},
\end{equation}

\emph{where $\Gamma(t)$ is the boundary of and $\vecx{N}$ denotes the inward unit normal vector of $\Gamma(t)$.}

\subparagraph{Derivation of the shape prior term} Applying the strategy described in \ref{sspar:GenFW_foulonneau} (with two levels of dependency) in order to minimize $J_{prior}$ leads, in the particular case \refeq{eq:NRJ_foulonneau2}, to the following flow:
\begin{equation}
  \label{eq:dphidt-foulonneau}
  \frac{\partial \Gamma}{\partial t} = \sum_{u,v}^{u+v\leq N} {A_{uv} \left( H_{uv}(x,y,\Omega_{int}) + \sum_{i=0}^2 { B_{uvi} \cdot L_i(x,y) } \right) } \vec{N},
\end{equation}
where
\begin{equation}
  \label{eq:Auv-foulonneau}
  A_{uv} = 2 \sum_{p,q}^{p+q\leq N} { \left( \lambda_{p,q} - \lambda_{p,q}^{ref} \right) C_{pq} a_{pu} a_{qv} },
\end{equation}
\begin{equation}
  \label{eq:Huv-foulonneau}
  H_{uv}(x,y,\Omega_{in}) = \frac{(x - \bar x)^u(y - \bar y)^v}{(\beta |\Omega_{in})^{(u+v+2)/2} },
\end{equation}
\begin{equation}
  \label{eq:Buv0-foulonneau}
  B_{uv0} = \frac{u \cdot \bar x}{\beta^\frac{1}{2} |\Omega_{in}|^\frac{3}{2}} \eta_{u-1,v} + \frac{v \cdot \bar y}{\beta^\frac{1}{2} |\Omega_{in}|^\frac{3}{2}} \eta_{u,v-1} - \frac{u + v + 2}{2|\Omega_{in}|} \eta_{u,v},
\end{equation}
\begin{equation}
  \label{eq:Buv1-2-foulonneau}
  B_{uv1} = -\frac{u}{\beta^\frac{1}{2} |\Omega_{in}|^\frac{3}{2}} \eta_{u-1,v}, \, \, \, \, \, \, \, \, \, \, \, \, \, \, \,
  B_{uv2} = -\frac{v}{\beta^\frac{1}{2} |\Omega_{in}|^\frac{3}{2}} \eta_{u,v-1},
\end{equation}
\begin{equation}
  \label{eq:L0-1-2-foulonneau}
  L_0 = 1, \, \, \, \, \, \, \, \, \, \, \, \, \, \, \,
  L_1 = x, \, \, \, \, \, \, \, \, \, \, \, \, \, \, \,
  L_2 = y.
\end{equation}
% --------------------- %
% A priori de mouvement %
% ---------------------%

\newpage
\chapter{\Ls~with motion prior}
\label{chap:mvt}

\section[Zhang and Pless]{Zhang and Pless 2005 \cite{Zhang2005}}
\label{sec:motion-zhang}

This method is designed for cardiopulmonary sequences segmentation. Thus the following framework only 2 degrees of freedom are used to parameterize the manifold.

\paragraph{Manifolds learning using isomap embedding}
~\par \vspace{0.3cm}
The Isomap procedure for dimensionality reduction starts by computing the distance between all pairs of images (using some distance function such as SSD pixel intensities). Then, a graph is defined with each image as a node and undirected edges connecting each image to its $k$-closest neighbors (usually choosing $k$ between $5$ and $10$). A complete pair-wise distance matrix is calculated by solving for the all-pairs shortest paths in this sparse graph. Finally, this complete distance matrix is embedded into some low dimension by solving an Eigenvalue problem (Multidimensional Scaling (MDS)). The dimensionality embedding can be chosen as desired, but ideally is the number of degrees of freedom in the image set.

For data sets with deformable motion, a suggested distance function is computed as the phase difference of local complex Gabor filters:
\begin{equation}
  \label{eq:local_Gabor_filter_zhang}
  \|I_1 - I_2 \|_{motion} = \sum_{x,y} { \psi(G(\omega, V, \sigma) \otimes I_1, G(\omega, V, \sigma) \otimes I_2) + \psi(G(\omega, H, \sigma) \otimes I_1, G(\omega, H, \sigma) \otimes I_2) }
\end{equation}
where $G(\omega, H, \sigma)$ is defined to be the 2D complex Gabor filter with frequency $\omega$, oriented either vertically or horizontally, with $\sigma$ as the variance of the modulating Gaussian, and $\psi$ returns the phase difference of the pair of complex Gabor responses above some threshold $\tau$.


\paragraph{Energy criterion}
~\par \vspace{0.3cm}
For cardiopulmonary image sequences, the images vary in principle depending on their cardiac phase $u$ and pulmonary phase $v$ - the two degrees of freedom that parameterize the manifold. Isomap is used to automatically parameterize all images, and interpolate the result to generate evenly spaced samples of the image manifold $f(x, y, u, v)$. Accordingly, the seeked contour $C$ is also a function of $u$ and $v$, and $C$ has to be described implicitly by the \ls~function $\phi$ in 4-dimension space $\Omega$. Thus, a given cardiopulmonary image sequence specifies this contour by extending the energy functional \refeq{eq:NRJ_chanvese} to 4-dimension space $\underset{c1,c2,\phi}{inf}E(c1,c2,\phi)$, where $c_1$ and $c_2$ are the interior and exterior means respectively and $\phi: \Rset^4 \rightarrow \Rset$.

But the manifold dimensions also correspond to specific kinds of deformation. The breathing of the patient results, approximately, in a translation of the heart. Therefore, the variation of $\phi$ in the $v$ direction is expected to be a uniform translation. That is, the energy functional change $\frac{\partial \phi}{\partial v}$ should be consistent with a uniform translation. This induces a \ls~corollary to the classic optic flow constraint equation:
\begin{equation}
  \label{eq:ls_OpticalFlow_zhang}
  \frac{\partial \phi}{\partial x}\omega_x + \frac{\partial \phi}{\partial y}\omega_y + \frac{\partial \phi}{\partial v} = 0
\end{equation}
where $(\omega_x, \omega_y)^T$ is the velocity vector that is constant over any given image, but may vary for different values of $u$ and $v$.

On the other hand, varying images along the other axis of the image manifold, deformations due to the cardiac cycle lead to image variation with minimal overall translation. For the special case of deformation caused by (non-uniform) heart expansion and contraction, the constraint can be expressed as:
\begin{equation}
  \label{eq:ls_dphidu_zhang}
  \frac{\partial \phi}{\partial u} = \omega_u
\end{equation}
where $\omega_u$ is constant over the region of the heart for any given $u$ and $v$. This constraint enforces the condition that moving along the ``heartbeat'' axis simply adds or subtracts a constant value of the \ls~function $\phi$, and therefore enforces that the shape either expands or shrinks.

Using these two constraints, the motion constraints can be written as an energy functional:
\begin{equation}
  \label{eq:NRJ_zhang}
  E(\phi) = \eta_1 \int_\Omega \left( \frac{\partial \phi}{\partial x}\omega_x + \frac{\partial \phi}{\partial y}\omega_y + \frac{\partial \phi}{\partial v} \right)^2 \, dxdy + \eta_2 \int_\Omega \left( \frac{\partial \phi}{\partial u} - \omega_u \right)^2 \, dxdy.
\end{equation}

\vspace{0.3cm}
\paragraph{Evolution equation}
~\par \vspace{0.3cm}
Given an initial \ls~function $\phi_0$, the functional \refeq{eq:NRJ_zhang} is minimized by iterating two steps, first using the current estimate of $\phi$ to estimate $c_1$, $c_2$ and solving for $\omega_x(u, v)$, $\omega_y(u, v)$, and $\omega_u(u, v)$, and then evolving $\phi$ by:
\begin{eqnarray}
  \label{eq:dphidt_zhang}
  \nonumber \frac{\partial \phi}{\partial t} = &\delta(\phi(\vecx{x})) \nabla_\phi F(I(\vecx{x}), \phi(\vecx{x})) + \lambda \delta(\phi(\vecx{x})) \text{div}\left(\frac{\nabla \phi(\vecx{x})}{\|\nabla \phi(\vecx{x})\|}\right) + 2\eta_2 \left( \dfrac{\partial^2 \phi}{\partial u^2} - \dfrac{\partial \omega_u}{\partial u} \right) \\
 & + 2\eta_1 \left( \dfrac{\partial^2 \phi}{\partial x^2}\omega_x^2 + \dfrac{\partial^2 \phi}{\partial y^2}\omega_y^2 + \dfrac{\partial^2 \phi}{\partial v^2} + 2\dfrac{\partial^2 \phi}{\partial x\partial y}\omega_x\omega_y + 2\dfrac{\partial^2 \phi}{\partial x\partial v}\omega_x + 2\dfrac{\partial^2 \phi}{\partial y\partial v}\omega_y \right)
\end{eqnarray}


\vspace{0.3cm}

\newpage
\section[Cremers and Soatto]{Cremers and Soatto 2005 \cite{Cremers2005, Cremers2003}}
\label{sec:motion-cremers}

\paragraph{Energy criterion}
~\par \vspace{0.3cm}
Let $\Omega \in \Rset^2$ denote the image plane and let $f: \Omega \times \Rset \rightarrow \Rset$ be a gray value image sequence. Denote the spatio-temporal image gradient of $f(\vecx{x}, t)$ by $\nabla_3 f = \left( \frac{\partial f}{\partial x_1}, \frac{\partial f}{\partial x_2}, \frac{\partial f}{\partial t} \right)^t$ and let $ v : \Omega \rightarrow \Rset^3, \, v(\vecx{x}) = ( u(\vecx{x}), w(\vecx{x}), 1)^t$ be the velocity vector at a point $\vecx{x}$ in homogeneous coordinates.

With these definitions, the problem of motion estimation now consists in maximizing the conditional probability
\begin{equation}
  \label{eq:Proba_v-gradf_cremers2005}
  P(v | \nabla_3 f) = \frac{P(\nabla_3 f | v)P(v)}{P(\nabla_3 f)}
\end{equation}
with respect to the motion field $v$.

Except for locations where the spatio-temporal gradient vanishes, the \emph{optic flow constraint} 
\begin{equation}
  \label{eq:optic_flow}
  \frac{df}{dt} = \frac{\partial f}{\partial t} + \frac{\partial f}{\partial x_1} \frac{dx_1}{dt} + \frac{\partial f}{\partial x_2} \frac{dx_2}{dt} = v^t \nabla_3f = 0
\end{equation}
states that the homogeneous velocity vector must be orthogonal to the spatio-temporal image gradient. Therefore a measure of this orthogonality is used as a conditional probability on the spatio-temporal image gradient:
\begin{equation}
  \label{eq:Proba_gradf-v_cremers2005}
  P(\nabla_3 f | v) \varpropto \exp \left( - \frac{( v(\vecx{x})^t \nabla_3 f(\vecx{x}) )^2}{ |v(\vecx{x})|^2 |\nabla_3 f(\vecx{x})|^2} \right)
\end{equation}

The velocity field $v$ is discretized by a set of disjoint regions $\Omega_i \subset \Omega$ with velocity $v_i$: $v(\vecx{x}) = {v_i,\text{ if }\vecx{x} \in \Omega_i}$. The prior probability on the velocity field is assumed to only depend on the length $L(C)$ of the boundary $C$ separating these regions:
\begin{equation}
  \label{eq:Proba_v_cremers2005}
  P(v) \varpropto \exp \left( - \mu L(C) \right)
\end{equation}

With the above assumptions, the first term in the numerator of \refeq{eq:Proba_v-gradf_cremers2005} can be written as:
\begin{equation}
  \label{eq:Proba_gradf-v2_cremers2005}
  P(\nabla_3 f | v) = \prod_{x \in \Omega} P(\nabla_3 f(\vecx{x}) | v(\vecx{x}))^h = \prod_{i=1}^n { \prod_{x \in \Omega_i} { P(\nabla_3 f(\vecx{x}) | v_i)^h } },
\end{equation}
where $h = d\vecx{x}$ denotes the pixel size of the discretization of $\Omega$.

Maximizing the conditional probability \refeq{eq:Proba_v-gradf_cremers2005} with respect to the velocity field $v$ is equivalent to minimize the negative logarithm of this expression, which is given by the energy functional:
\begin{equation}
  \label{eq:NRJ_C_cremers2005}
  E(C,\lbrace v_i \rbrace) = \sum_{i=1}^n {\int_{\Omega_i} \frac{( v(\vecx{x})^t \nabla_3 f(\vecx{x}) )^2}{ |v(\vecx{x})|^2 |\nabla_3 f(\vecx{x})|^2} \, d\vecx{x} + \mu L(C) }.
\end{equation}
Using the \ls~function $\phi$ (for a two-phase segmentation), eq.\refeq{eq:NRJ_C_cremers2005} can be rewritten:
\begin{equation}
  \label{eq:NRJ_phi_cremers2005}
  E(\phi,v_1, v_2) = \int_\Omega \frac{( v_1(\vecx{x})^t \nabla_3 f(\vecx{x}) )^2}{ |v_1(\vecx{x})|^2 |\nabla_3 f(\vecx{x})|^2} H(\phi(\vecx{x})) \, d\vecx{x} + \int_\Omega \frac{( v_2(\vecx{x})^t \nabla_3 f(\vecx{x}) )^2}{ |v_2(\vecx{x})|^2 |\nabla_3 f(\vecx{x})|^2} (1-H(\phi(\vecx{x}))) \, d\vecx{x} + \mu L(C).
\end{equation}


\paragraph{Velocity field expression}
~\par \vspace{0.3cm}
In order to cope with complex motion regions, piecewise parametric motion can be used. The velocity on the domain $\Omega_i$ is allowed to vary according to a model of the form:
\begin{equation}
  \label{eq:v-model_cremers2005}
  v_i(\vecx{x}) = M(\vecx{x})p_i,
\end{equation}
where $M$ is a matrix depending only on space and time and $p_i$ is the parameter vector associated with each region.

Inserting model \refeq{eq:v-model_cremers2005} into the \emph{optic flow constraint} \refeq{eq:optic_flow} gives a relation which states that the vector $M^t \nabla_3 f$ must either vanish or be orthogonal to the vector $p_i$. The energy functional can thus be rewritten as:
\begin{equation}
  \label{eq:NRJ2_phi_cremers2005}
  E(\phi,p_1, p_2) = \int_\Omega \frac{p_1^t T(\vecx{x})p_1}{|p_1|^2} H(\phi(\vecx{x})) \, d\vecx{x} + \int_\Omega \frac{( p_2^t T(\vecx{x})p_2 )^2}{|p_2|^2} (1-H(\phi(\vecx{x}))) \, d\vecx{x} + \mu L(C),
\end{equation}
where $T(\vecx{x}) = \dfrac{ \nabla_3 f^t(\vecx{x}) M(\vecx{x}) M^t(\vecx{x}) \nabla_3 f(\vecx{x}) }{ |M^t(\vecx{x}) \nabla_3 f(\vecx{x})|^2 }$ (\todo{$T(\vecx{x}) = \dfrac{ M(\vecx{x}) \nabla_3 f^t(\vecx{x}) \nabla_3 f(\vecx{x}) M^t(\vecx{x}) }{ |M^t(\vecx{x}) \nabla_3 f(\vecx{x})|^2 }$??}).


\paragraph{Evolution equation}
~\par \vspace{0.3cm}
For fixed motion vectors, the gradient descent on the functional \refeq{eq:NRJ_phi_cremers2005} for the \ls~function $\phi$ is given by:
\begin{equation}
  \label{eq:dphidt_cremers2005}
  \frac{\partial \phi}{\partial t} = \delta(\phi(\vecx{x})) \left[ \mu \cdot \text{div} \left( \frac{\nabla \phi}{\| \nabla \phi \|} \right) + e_2 - e_1 \right]
\end{equation}
with the energy densities $e_i$ given by:
\begin{equation}
  \label{eq:ei_cremers2005}
  e_i = \frac{p_i^t T(\vecx{x}) p_i}{p_i^t p_i} = \frac{p_i^t \nabla_3 f^t(\vecx{x}) M(\vecx{x}) M^t(\vecx{x}) \nabla_3 f(\vecx{x}) p_i}{|p_i|^2 |M^t(\vecx{x}) \nabla_3 f(\vecx{x})|^2}
\end{equation}


For fixed $\phi$, minimization of the functional \refeq{eq:NRJ_phi_cremers2005} with respect to the motion vectors $p_1$ and $p_2$ results in the eigenvalue problem:
\begin{equation}
  \label{eq:dpdt_cremers2005}
  p_i = \underset{p}{argmin} \left( \frac{p^t T_i(\vecx{x}) p}{p^t p} \right),
\end{equation}
for the matrices
\begin{eqnarray}
  \label{eq:Ti_cremers2005}
  T_1 = \int_\Omega T(\vecx{x})H(\phi((\vecx{x})) \, d\vecx{x} \text{ \hspace*{0.3cm} and}\\
  T_2 = \int_\Omega T(\vecx{x})(1-H(\phi((\vecx{x}))) \, d\vecx{x}
\end{eqnarray}

The solution of \refeq{eq:dpdt_cremers2005} is given by the eigenvectors corresponding to the smallest eigenvalues of $T_1$ and $T_2$, normalized such that its last component is 1.


\newpage
\cite{Papin2000}

% -- End of the declaration of each part
%___________________________________________________________
%

\newpage

\pagestyle{fancy}
\fancyhead{} % Headers
\fancyhead[LE,RO]{\nouppercase\leftmark}

\bibliography{src/bibliographie}
\bibliographystyle{IEEEbib}
\addcontentsline{toc}{chapter}{Bibliography}

\end{document}